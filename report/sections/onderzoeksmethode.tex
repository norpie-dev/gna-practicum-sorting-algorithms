\section{Onderzoeksmethode}

De sorteeralgoritmen zijn in de Java programmeertaal geschreven volgens hun standaardversies.

Om de algemene efficiëntie van de sorteeralgoritmen te meten werden er drie soorten invoerrijen in de algoritmen gevoert: willekeurige invoerrijen, beste-geval invoerrijen, en slechst-geval invoerrijen. Elke invoer werd 50 keer getest om kleine een zo groot mogelijke steekproefomvang te hebben. Elke invoer grootte tussen 0 en 101 werden getest.

Het doubling-ratio van InsertionSort en QuickSort werd gemeten door willekeurige invoerrijen te genereren voor een bepaalde grootte N. Deze N werd dan gedubbeld en opnieuw gemeten. Dit werd 6 keer in totaal gedaan, waarna er een voorspelling wordt gemaakt voor de volgende 3 grootte N gebaseerd op de data.
